\documentclass[usenames,dvipsnames, 9pt]{beamer}
\usepackage{amsmath,amsfonts,amssymb}
\usepackage{mathtools}
\usepackage{etex} %for Windows
\usepackage[utf8]{inputenc}
\usepackage[english, russian]{babel} 
%\usepackage{microtype}			% Better interword spacing and additional kerning.
\usepackage{ellipsis}			% Adjusted space with \dots between two words.
\usepackage{graphicx}
\usepackage{pstricks}

\usepackage{xcolor}


\usepackage{changepage}

\usepackage{algorithm}
\usepackage{algpseudocode}
%\usepackage[]{algorithm2e}
%\usepackage{algorithmic}

%\usepackage{tcolorbox}


\usepackage{caption}
\usepackage{subcaption}
%\usepackage{stackengine}


\usepackage{tikz}
\usetikzlibrary{tikzmark,calc}
\usetikzlibrary{positioning, backgrounds}
\usetikzlibrary{arrows, chains, matrix, scopes, patterns, shapes, fit}
\usetikzlibrary{mindmap,trees,shadows}
\usetikzlibrary{decorations.pathreplacing}
%\usetikzlibrary{crypto.symbols}

\usepackage{pgfplots}

\pgfmathdeclarefunction{gauss}{2}{%
	\pgfmathparse{1/(#2*sqrt(2*pi))*exp(-((x-#1)^2)/(2*#2^2))}%
}


\tikzset{
	invisible/.style={opacity=0},
	visible on/.style={alt={#1{}{invisible}}},
	alt/.code args={<#1>#2#3}{%
		\alt<#1>{\pgfkeysalso{#2}}{\pgfkeysalso{#3}} % \pgfkeysalso doesn't change the path
	},
}

\newcommand\strikeout[2][]{%
	\begin{tabular}[b]{@{}c@{}} 
		\makebox(0,0)[cb]{{#1}} \\[-0.2\normalbaselineskip]
		\rlap{\color{Orange}\rule[0.5ex]{\widthof{#2}}{1.5pt}}#2
\end{tabular}}

\newcommand\Fontvi{\fontsize{11}{13.2}\selectfont}

\usepackage{listings} % for C++ code

\usepackage{braket}
%\usepackage[braket, qm]{qcircuit}



\usepackage[T1]{fontenc}
%\usepackage[sfdefault,scaled=.85]{FiraSans}
%\usepackage{newtxsf}
%\usepackage[nomap]{FiraMono}





\usefonttheme[onlymath]{serif}
\renewcommand\sfdefault{cmbr}

\renewcommand{\bfdefault}{sb}

\definecolor{CharCoalDark}{RGB}{13, 16, 19}
\definecolor{Orange}{RGB}{255, 165,0}
\definecolor{DarkOrange}{RGB}{255, 165,0}
\definecolor{LightSalmon}{RGB}{255, 160, 122}
\definecolor{LeafGreen}{RGB}{34, 139,  34}
\definecolor{Coral}{RGB}{255, 127, 80}
\definecolor{DarkTurquoise}{RGB}{0, 206, 209}

%\newtheorem{defRus}{Определение}
%\newtheorem{thmRus}{Теорема}
%s\newtheorem{corRus}{Следствие}


\setbeamercolor{background canvas}{bg=CharCoalDark}

\setbeamerfont{title}{series=\bfseries}
\setbeamercolor{title}{fg=Orange}
\setbeamercolor{section in toc}{fg=white}
\setbeamercolor{frametitle}{fg=Orange}
\setbeamercolor{normal text}{fg=white}
%\setbeamercolor{normal text}{fontsize=12pt}
\setbeamercolor{itemize item}{fg=Orange}
\setbeamercolor{enumerate item}{fg=Orange}
\setbeamercolor{enumerate item item}{fg=Orange}
\setbeamercolor{itemize item item}{fg=Orange}
\setbeamercolor{enumerate item}{fg=Orange}
\setbeamercolor{block title}{bg=DarkOrange,fg=white}
\setbeamerfont{block title}{series=\bfseries}

\setbeamertemplate{itemize item}[circle]
\setbeamertemplate{eumerate subitem}{\color{Orange}[$\checkmark$]}
\setbeamertemplate{itemize subitem}{\color{Orange}\Large$\textbullet$}
\setbeamertemplate{itemize subitem}{\color{Orange} \tiny $\blacksquare$}

% footnote without a marker
\newcommand\blfootnote[1]{%
	\begingroup
	\renewcommand\footnoterule{}
	\renewcommand\thefootnote{}\footnote{#1}%
	\addtocounter{footnote}{-1}%
	\endgroup
}

\newcommand*{\Scale}[2][4]{\scalebox{#1}{\ensuremath{#2}}}%

\newcommand\Item[1][]{%
	\ifx\relax#1\relax  \item \else \item[#1] \fi
	\abovedisplayskip=0pt\abovedisplayshortskip=0pt~\vspace*{-\baselineskip}}

\pgfdeclareradialshading{ring}{\pgfpoint{0cm}{0cm}}%
{rgb(0cm)=(1,1,1);
	rgb(0.7cm)=(1,1,1);
	rgb(0.719cm)=(1,1,1);
	rgb(0.72cm)=(0.975,0,0);
	rgb(0.9cm)=(1,1,1)}

\usepackage[absolute,overlay]{textpos} %to clip to a corner
\newcommand\FrameText[1]{%
	\begin{textblock*}{\paperwidth}(\textwidth-35pt, 10 pt)
		\raggedright #1\hspace{.5em}
\end{textblock*}}

\makeatletter
\let\save@measuring@true\measuring@true
\def\measuring@true{%
	\save@measuring@true
	\def\beamer@sortzero##1{\beamer@ifnextcharospec{\beamer@sortzeroread{##1}}{}}%
	\def\beamer@sortzeroread##1<##2>{}%
	\def\beamer@finalnospec{}%
}
\makeatother

\AtBeginSection[]
{
	\begin{frame}<beamer>
		\frametitle{Outline}
		\tableofcontents[currentsection]
	\end{frame}
}


%\institute{ENS Lyon}
\author{Elena Kirshanova \\ [10pt]
}
\titlegraphic{
	
	\includegraphics[width=2.5cm]{stayhome}%
	%\includegraphics[width=4.0cm]{ens_logo_gray}
}
\title{\Huge Decentralized Privacy-Preserving Proximity Tracing \\[5pt]
\Large Based on the document \url{https://github.com/DP-3T/documents}
}

\date{ Course ``Information and Network Security''  \today }


\setbeamertemplate{navigation symbols}{} %removes navigation

% proper highlightling of a code-snippet
\lstset{language=C++,
	keywordstyle=\color{magenta},
	stringstyle=\color{Goldenrod},
	commentstyle=\color{gray},
	breaklines=false,
	%morecomment=[l][\color{magenta}]{\#}
}

%\setlength{\parskip}{8pt}
\input{header} %all defs
\begin{document}
	
\begin{frame}
	\titlepage
\end{frame}

\begin{frame}{Goal}
	\Large 
		{\color{Orange}  How to identify people who have been in contact with an infected person? }\\[5pt]
	\pause
	Academics issued a proposal of a design of an app aiming at 	{\color{Orange} proximiti tracing} -- determining who has been in close physical proximity to an infected person.\\[10pt]
	
	Main purpose: quick notification of contact people \\[10pt]
	
	{\color{Orange} Ideally} the app should ensure {\color{Orange}security} and {\color{Orange} privacy} for users.  \\
	
	This mini-lecture: overview of the design proposal \\
	
	
	\vfill
	Please visit \url{https://github.com/DP-3T/documents} for  details
	
\end{frame}

\begin{frame}{High level idea}
	
\end{frame}


\begin{frame}{High level idea}
\begin{itemize}
	\item the app generates a pseudorandom ID (using a PRG) that represents a user 
	\item This ID is continually broadcasted via BlueTooth
	\item the app records all IDs observed in the proximity
	\item if ID$^\star$ is diagnosed for the virus, the user ID$^\star$ uploads all its recorded IDs to a central server
	\item  the app once in a while downloads data from the severs and locally checks if any of the recorded IDs has been diagnosed for the virus
\end{itemize}
\pause
{\color{Orange} Privacy} considerations:
\begin{itemize}
\item the central server only stores IDs of infected people (no proximity data is uploaded)
\item Nobody can track non-infected users seeing only their IDs
\item No entity should be able to use the known data for  malicious purposes (e.g.,  misuse of recording)
\end{itemize}
\end{frame}

\begin{frame}{More details I}
%\large
\begin{enumerate}
	\itemsep 1em
	\item  {\color{Orange}Secret Key Generation:}
	
	For the day $t$, the smartphone generates a random key $S_t $:
	
	\[S_t \leftarrow \KeyGen(t)\]
	 using, for example, \ a PRG (see lecture 2). \\[5pt]
	 
	 For the next days, the secret keys $S_{t+i}$ are generated using a {\color{Orange}cryptographic hash function} $\Hash$:
	 \[
	 	S_{t+i} = \Hash(S_{t+i-1})
	 \]
	 \pause 
	 \item {\color{Orange} Ephemeral ID generation:}  Each user changes his $\EID$ $n$ times per day (tunable parameter). \\
	 At the beginning of day $t$, the app generates $n$ ephemeral IDs  $\EID$ as
	 
	 \[
	 	\EID_1 || \ldots || \EID_n = \text{PRG} (  \Hash (S_t, \texttt{``broadcast key''}) ),
	 \]
	 \pause 
	 Each $|\EID_i|$ is 16 bytes (Bluetooth Low Energy beacons payload)
	 
\end{enumerate}
\end{frame}

\begin{frame}{More details II}
\begin{enumerate}
	\itemsep 1em
	\setcounter{enumi}{2}
	\item {\color{Orange} Local storage of observed data:}
	``Seeing'' any $\EID$ in proximity, each smartphone stores
	\begin{itemize}
		\item $\EID$ 
		\item duration of contact
		\item proximity
		\item time indication (April, 13)
	\end{itemize} 
This whole data occupies approx.\ 32 bytes.
\pause
	\item {\color{Orange} If a user is infected:} on the day $t$:	
	\begin{itemize}
	\item The user sends $S_t$ to the central server 
	\item The sever stores pairs $(S_t, t)$ and broadcasts them to all users.
	\item The infected user chooses a completely new secret key 
	\end{itemize}
\pause
\item {\color{Orange} Notification of risk:} 	\\
	\begin{itemize}
		\item The server broadcasts secret keys $(S_t,t)$ of infected users
		\item Each user loads $(S_t,t)$  and recomputes the corresponding $\EID'$s
		\item And checks these $\EID'$ against the locally stored ones. 
	\end{itemize}
\end{enumerate}
\end{frame}

{
	\setbeamercolor{background canvas}{bg=white}
	
	\begin{frame}{{\color{Black}In picture I}}
	\vspace{-30pt}
	\begin{figure}
		\includegraphics[width=1.1\textwidth]{Pic1}
	\end{figure}
	\small
	\vfill
	{\color{gray} The picture is taken from DP3T White Paper.pdf \url{https://github.com/DP-3T/documents/blob/master/DP3T\%20White\%20Paper.pdf}}
\end{frame}
	
	\begin{frame}{{\color{Black}In picture II}}
	\vspace{-30pt}
		\begin{figure}
			\includegraphics[width=1.1\textwidth]{Pic2}
		\end{figure}
	\small
	\vspace{-10pt}
	{\color{gray} The picture is taken from DP3T White Paper.pdf \url{https://github.com/DP-3T/documents/blob/master/DP3T\%20White\%20Paper.pdf}}
\end{frame}
}

\end{document}
