\documentclass[usenames,dvipsnames, 9pt]{beamer}
\usepackage{amsmath,amsfonts,amssymb}
\usepackage{mathtools}
\usepackage{etex} %for Windows
\usepackage[utf8]{inputenc}
\usepackage[english, russian]{babel} 
%\usepackage{microtype}			% Better interword spacing and additional kerning.
\usepackage{ellipsis}			% Adjusted space with \dots between two words.
\usepackage{graphicx}
\usepackage{pstricks}

\usepackage{xcolor}


\usepackage{changepage}

\usepackage{algorithm}
\usepackage{algpseudocode}
%\usepackage[]{algorithm2e}
%\usepackage{algorithmic}

%\usepackage{tcolorbox}


\usepackage{caption}
\usepackage{subcaption}
%\usepackage{stackengine}


\usepackage{tikz}
\usetikzlibrary{tikzmark,calc}
\usetikzlibrary{positioning, backgrounds}
\usetikzlibrary{arrows, chains, matrix, scopes, patterns, shapes, fit}
\usetikzlibrary{mindmap,trees,shadows}
\usetikzlibrary{decorations.pathreplacing}
%\usetikzlibrary{crypto.symbols}

\usepackage{pgfplots}

\pgfmathdeclarefunction{gauss}{2}{%
	\pgfmathparse{1/(#2*sqrt(2*pi))*exp(-((x-#1)^2)/(2*#2^2))}%
}


\tikzset{
	invisible/.style={opacity=0},
	visible on/.style={alt={#1{}{invisible}}},
	alt/.code args={<#1>#2#3}{%
		\alt<#1>{\pgfkeysalso{#2}}{\pgfkeysalso{#3}} % \pgfkeysalso doesn't change the path
	},
}

\newcommand\strikeout[2][]{%
	\begin{tabular}[b]{@{}c@{}} 
		\makebox(0,0)[cb]{{#1}} \\[-0.2\normalbaselineskip]
		\rlap{\color{Orange}\rule[0.5ex]{\widthof{#2}}{1.5pt}}#2
\end{tabular}}

\newcommand\Fontvi{\fontsize{11}{13.2}\selectfont}

\usepackage{listings} % for C++ code

\usepackage{braket}
%\usepackage[braket, qm]{qcircuit}



\usepackage[T1]{fontenc}
%\usepackage[sfdefault,scaled=.85]{FiraSans}
%\usepackage{newtxsf}
%\usepackage[nomap]{FiraMono}





\usefonttheme[onlymath]{serif}
\renewcommand\sfdefault{cmbr}

\renewcommand{\bfdefault}{sb}

\definecolor{CharCoalDark}{RGB}{13, 16, 19}
\definecolor{Orange}{RGB}{255, 165,0}
\definecolor{DarkOrange}{RGB}{255, 165,0}
\definecolor{LightSalmon}{RGB}{255, 160, 122}
\definecolor{LeafGreen}{RGB}{34, 139,  34}
\definecolor{Coral}{RGB}{255, 127, 80}
\definecolor{DarkTurquoise}{RGB}{0, 206, 209}

%\newtheorem{defRus}{Определение}
%\newtheorem{thmRus}{Теорема}
%s\newtheorem{corRus}{Следствие}


\setbeamercolor{background canvas}{bg=CharCoalDark}

\setbeamerfont{title}{series=\bfseries}
\setbeamercolor{title}{fg=Orange}
\setbeamercolor{section in toc}{fg=white}
\setbeamercolor{frametitle}{fg=Orange}
\setbeamercolor{normal text}{fg=white}
%\setbeamercolor{normal text}{fontsize=12pt}
\setbeamercolor{itemize item}{fg=Orange}
\setbeamercolor{enumerate item}{fg=Orange}
\setbeamercolor{enumerate item item}{fg=Orange}
\setbeamercolor{itemize item item}{fg=Orange}
\setbeamercolor{enumerate item}{fg=Orange}
\setbeamercolor{block title}{bg=DarkOrange,fg=white}
\setbeamerfont{block title}{series=\bfseries}

\setbeamertemplate{itemize item}[circle]
\setbeamertemplate{eumerate subitem}{\color{Orange}[$\checkmark$]}
\setbeamertemplate{itemize subitem}{\color{Orange}\Large$\textbullet$}
\setbeamertemplate{itemize subitem}{\color{Orange} \tiny $\blacksquare$}

% footnote without a marker
\newcommand\blfootnote[1]{%
	\begingroup
	\renewcommand\footnoterule{}
	\renewcommand\thefootnote{}\footnote{#1}%
	\addtocounter{footnote}{-1}%
	\endgroup
}

\newcommand*{\Scale}[2][4]{\scalebox{#1}{\ensuremath{#2}}}%

\newcommand\Item[1][]{%
	\ifx\relax#1\relax  \item \else \item[#1] \fi
	\abovedisplayskip=0pt\abovedisplayshortskip=0pt~\vspace*{-\baselineskip}}

\pgfdeclareradialshading{ring}{\pgfpoint{0cm}{0cm}}%
{rgb(0cm)=(1,1,1);
	rgb(0.7cm)=(1,1,1);
	rgb(0.719cm)=(1,1,1);
	rgb(0.72cm)=(0.975,0,0);
	rgb(0.9cm)=(1,1,1)}

\usepackage[absolute,overlay]{textpos} %to clip to a corner
\newcommand\FrameText[1]{%
	\begin{textblock*}{\paperwidth}(\textwidth-35pt, 10 pt)
		\raggedright #1\hspace{.5em}
\end{textblock*}}

\makeatletter
\let\save@measuring@true\measuring@true
\def\measuring@true{%
	\save@measuring@true
	\def\beamer@sortzero##1{\beamer@ifnextcharospec{\beamer@sortzeroread{##1}}{}}%
	\def\beamer@sortzeroread##1<##2>{}%
	\def\beamer@finalnospec{}%
}
\makeatother

\AtBeginSection[]
{
	\begin{frame}<beamer>
		\frametitle{Outline}
		\tableofcontents[currentsection]
	\end{frame}
}


%\institute{ENS Lyon}
\author{Elena Kirshanova \\ [10pt]
}
\titlegraphic{
	
	%\includegraphics[width=2.5cm]{stayhome}%
	%\includegraphics[width=4.0cm]{ens_logo_gray}
}
\title{\Huge Authenticated Encryption}

\date{ Course ``Information and Network Security'' \\ 	
	Lecture 7 \\ \today }


\setbeamertemplate{navigation symbols}{} %removes navigation

% proper highlightling of a code-snippet
\lstset{language=C++,
	keywordstyle=\color{magenta},
	stringstyle=\color{Goldenrod},
	commentstyle=\color{gray},
	breaklines=false,
	%morecomment=[l][\color{magenta}]{\#}
}

%\setlength{\parskip}{8pt}
\input{header} %all defs
\begin{document}
	
\begin{frame}
	\titlepage
\end{frame}

\begin{frame}{Agenda}
	\Large 
	Up until now: 
	\begin{itemize}
		\item Confidentiality (using Symmetric Encryption)
		\item Integrity (MAC, HMAC)
	\end{itemize}
	\vspace{20pt}
	These were protections against {\color{Orange} eavesdropping (\textbf{passive} adversary)}\\
	\pause 
	\vspace{20pt}
	Today: \\
	Protect data against {\color{Orange} (tampering) (\textbf{active} adversary)}:\\
	\LARGE Authenticated Encryption

\end{frame}

\begin{frame}{Authenticated Encryption: definition}
\Large

An {\color{Orange}{Authenticated Encryption (AE)}} system consists of three ppt algorithms
\begin{itemize}
	\itemsep 10pt
	\item Key generation: $\KeyGen(1^\lambda): k \leftarrow \keyS$
	\item Encryption: $\Enc: \keyS \times \mesS \times \nonceS \rightarrow \cipS$
	\item Decryption:  $\Dec: \keyS \times \cipS  \times \nonceS \rightarrow \mesS \cup \{\bot\}$
\end{itemize}
\vspace{15pt}
$\keyS$ - key space, $\mesS$ - message space, $\cipS$ - ciphertext space, $\nonceS$ - {\color{Orange} nonce} space. \\[10pt]
\pause

NEW: $\{\bot\}$ - ciphertext is rejected \\


\pause
{\color{Orange} Nonce} = ``number that can only be used once'' \\
It can be predictable, but should {\color{Orange} never be used  twice} for the same key. \\
Example: values derived from IV in various  modes of encryption. 

\end{frame}

\begin{frame}{Security of AE}
\Large
An { \color{Orange}{Authenticated Encryption (AE)}} system consists of three ppt algorithms
\begin{itemize}
	\itemsep 10pt
	\item Key generation: $\KeyGen(1^\lambda): k \leftarrow \keyS$
	\item Encryption: $\Enc: \keyS \times \mesS \times \nonceS \rightarrow \cipS$
	\item Decryption:  $\Dec: \keyS \times \cipS \times \mesS \times \nonceS \rightarrow \mesS \cup \{\bot\}$
\end{itemize}

\vspace{15pt}
{\color{Orange}{Correctness:}} $\forall m, \forall k, \forall n:$ $\Dec(k, \Enc(k, m,n), n) = m$
\pause
\vspace{15pt}
{\color{Orange}{Security:}}
\begin{itemize}
	\itemsep 10pt
	\item $\Enc(k, m_0, n)$ is indistinguishable from $\Enc(k, m_1, n)$ $\forall m_0 != m_1$ (without knowledge of $k$) 
	\item No ppt adversary can create a new ciphertext that does not decrypt to $\{\bot\}$.
\end{itemize}

\end{frame}

\begin{frame}{Security of AE}
	\Large
	Authenticated Encryption provides
	\begin{itemize}
		\itemsep 10pt
		\item {\color{Orange}Authenticity:} If $\Dec(k, c, n)!=\{\bot\}$, then the receiver is ensured that the message comes from someone who knows $k$
		\pause
		\item {\color{Orange}AE $\implies$ Chosen Ciphertext Security} 
	\end{itemize}
\pause
\vspace{15pt}
In Chosen Ciphertext Attack (CCA) an adversary can
\begin{itemize}
	\item obtain encryptions of messages of his choice
	\item ask for decryption of {\color{Orange}{any}} ciphertext of his choice except one specific ``challenge'' $c$
\end{itemize}
\vfill
\centering
\color{Orange} A CCA adversary is a very powerful adversary. \\
Why does it capture real life attacks?
\end{frame}

\begin{frame}{Example of CCA attack (IPSec, simplified)}
\Large 
Let $\Enc$ be a block cipher in CTR mode \\
The message $m$ consists of a header ``to Bob''+the rest 

\LARGE
	\begin{center}
		\begin{tabular}{c c c c c}
			Alice&     &Mail server  & &Bob\\
			$k$ & {\huge $\xrightarrow{c \;= \Enc(k,\, m = \text{To Bob||...})}$ }  &  $k$ & $\xrightarrow{\hspace{10pt}m\hspace{10pt}}$ & \pause\\[8pt] 
			&  $\xrightarrow{\Huge \hspace{1em}c\hspace{1em}}$ Eve  $\xrightarrow{\hspace{1em}\hat{c}\hspace{1em}}$  & & & \\
		\end{tabular}
	\end{center}
\pause
\vspace{4pt}
\Large
Assume $\mathtt{len}$(``to Bob'') $==$  $\mathtt{len}$(``to Eve'') $==$ block-size.
\vspace{15pt}
\begin{columns}
	\begin{column}{0.45\linewidth}
		\begin{figure}
			\includegraphics[width=0.95\linewidth]{CTR}
		\end{figure}
	\end{column}
	\begin{column}{0.55\linewidth}
{\color{Orange} $\hat{c_1} = c_1 \oplus $[``to Bob''] $\oplus$ [``to Eve''] } \\[5pt]
The rest blocks  of $\hat{c}$ are equal to $c$.\\[5pt]
Eve knows $m$ by querying $\Dec(\hat{c})$.
	\end{column}

\end{columns}
\end{frame}

\begin{frame}{Construction of AE}
\Large 
\centering
{\color{Orange} AE = Secure Encryption + Secure Mac} \\[10pt]

Two keys: Encryption key $k_E$, MAC key $K_M$ \\
\pause 
\begin{center}
	Two main paradigms:
\end{center}
\vspace{10pt}

\begin{columns}
	\begin{column}{0.5\linewidth}
	{\color{Orange} I. Encrypt-then-MAC}
	\begin{enumerate}
		\item $c = \Enc(k_E, m)$
		\item $t = \MAC(k_M, c)$
		\item return $(c,t)$
	\end{enumerate}
Example: IPSec 
	\end{column} \pause
	\begin{column}{0.5\linewidth}
	 {\color{Orange} II. MAC-then-Encrypt}
		\begin{enumerate}
			\item $t = \MAC(k_M, n) $
			\item $c = \Enc(k_E, m||t)$
			\item return $c$
		\end{enumerate}
		Example: SSL 
\end{column}
\end{columns}

\vspace{10pt}

\pause
\begin{itemize}
	\item  Encrypt-then-MAC always provides AE
	\item MAC-then-Encrypt provides AE when $\Enc$ is randomized CTR/CBC mode encryption
	\item Other combinations of Mac and Encryption usually do not provide secure AE
\end{itemize}

\end{frame}

\begin{frame}{AE standards}
\LARGE
\begin{enumerate}
	\itemsep 10 pt
	\item \textbf{GCM (Galois Counter Mode)}. {\Large {\color{Orange}Encrypt-then-MAC}  \\
		Encryption: CTR mode + fast Mac (Carter-Wegman Mac).  \\
		Application: TLS \\
		Advantages: somewhat fast (on Intel)
	 }
 \pause
 	\item \textbf{CCM}.  {\Large {\color{Orange}MAC-then-Encrypt}  \\
 		Encryption: CBC MAC (AES)+ CTR mode (AES)  \\
 		Application: 802.11i \\
 		Advantages: less code
 	}
 \pause
 	\item \textbf{ChaCha20-Poly1305.} {\Large {\color{Orange} Encrypt-then-MAC}  \\
 		Encryption: ChaCha20 Encryption + Poly1305 MAC \\
 		Application: TLS \\
 		Advantages: fast
 	}
\end{enumerate}
\pause
\vspace{10pt}
These three are implemented in OpenSSL. \\
I do not know of Russian AE standards (although one can replace Enc and MAC by Russian GOSTs).
\end{frame}

\begin{frame}{AEAD: Authenticated Encryption with Associated Data }
	\Large
	Often not all data needs to be encrypted.\\
	
	\LARGE
	\[
	\underbrace{
	\left[
	\texttt{Associated data} ||
	\texttt{Encrypted data}
	\right]
	}_\text{Authenticated}
	\]
	
	\vspace{25pt}
	
	Example: $[\texttt{header} || \texttt{payload} ]$ in internet protocols\\
	
	\vspace{25pt}

	Most used AEAD: {\color{Orange}AES-GCM AEAD}
	
\end{frame}

\begin{frame}{AES-GCM AEAD}
\Large Message $m=(m_1, \ldots, m_s)$
\only<1>
{
	\begin{figure}
		\includegraphics[width=\linewidth]{AES_GCM_AEAD_1}
	\end{figure}
}
\only<2>
{
	\begin{figure}
		\includegraphics[width=\linewidth]{AES_GCM_AEAD_2}
	\end{figure}
}
\only<3>
{
	\begin{figure}
		\includegraphics[width=\linewidth]{AES_GCM_AEAD_3}
	\end{figure}
}
\only<4->
{
	\begin{figure}
		\includegraphics[width=\linewidth]{AES_GCM_AEAD_full}
	\end{figure}
}
\only<5>
{
Output $(c_1, \ldots, c_s, c_{s+1})$
}
\end{frame}

\begin{frame}{AES-GCM AEAD}
\begin{figure}
	\includegraphics[width=0.7\linewidth]{AES_GCM_AEAD_full}
\end{figure}

\begin{itemize}
	\item Uses just one key
	\item MAC: GHASH (Galois Hash) - uses finite field arithmetic (fast)
	\item Decryption: \\
	1. Verifies MAC \\
	2. $\Dec(c_1, \ldots, c_s)$
\end{itemize}

\end{frame}
\begin{frame}{AEAD in TLS 1.3}
	\LARGE
	\begin{center}
		\begin{tabular}{c c c }
			\texttt{Browser}&  {\color{Orange}{Phase 1} Handshake}   & \texttt{Web server}  \\
			& {\Large Asymmetric Encryption}  & \\ 
			& {\Large Common keys are derived}  & \\ 
			$k_{b\rightarrow s}$&  & $k_{b\rightarrow s}$\\ 
			$k_{s\rightarrow b}$&  & $k_{s\rightarrow b}$ \pause \\ 
			&  {\color{Orange}{Phase 2} TLS record protocol}   &  \\
			& {\Large AEAD}  & \\ 
		\end{tabular}
	\end{center}
\end{frame}

\begin{frame}{TLS record protocol}
\Large Data = $\left[m_1, \ldots, m_s\right]$
\LARGE
\begin{center}
	\begin{tabular}{c c c }
\texttt{Browser}&   & \texttt{Web server}  \\
$k_{b\rightarrow s}$&  & $k_{b\rightarrow s}$\\ 
$k_{s\rightarrow b}$&  & $k_{s\rightarrow b}$  \\ 
$ctr_{b\rightarrow s}$&  & $ctr_{b\rightarrow s}$\\ 
$ctr_{s\rightarrow b}$& {\large$\underbrace{
		\left[
		\texttt{Meta data} || \;
		m_i \; || \;
		\texttt{Nonce}
		\right]}$}  & $ctr_{s\rightarrow b}$  \\ 
\multicolumn{3}{c}{ $\xrightarrow{ \hspace{1em}  \Huge \text{AES-GCM-AEAD}(k_{b\rightarrow s}) \hspace{3.4em} }$  }  \pause \\
$ctr_{b\rightarrow s}++$&  & $ctr_{b\rightarrow s}++$\\ 
\end{tabular}
\end{center}

\pause 
\vspace{10pt}
\texttt{Meta data} includes: record on the phase (1 or 2), TLS Version, \texttt{len}$(c)$ \\
Counters $ctr$ are used to prevent replay attacks

\end{frame}

\begin{frame}{Programming Assignment \# 5}
\Large
OpenSSL provides interfaces to GCM, CCM AEs via \texttt{EVP}\\[15pt]


This PA:  to implement Encryption and Decryption Interfaces for any two Authenticated Encryption
\begin{itemize}
	\item GCM 
	\item CCM 
	\item ChaCha20-Poly1305
\end{itemize}

\vspace{10pt}
See \url{https://wiki.openssl.org/index.php/EVP_Authenticated_Encryption_and_Decryption} for code


\end{frame}

\end{document}
