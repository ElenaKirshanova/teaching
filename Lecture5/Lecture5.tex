\documentclass[usenames,dvipsnames, 9pt]{beamer}
\usepackage{amsmath,amsfonts,amssymb}
\usepackage{mathtools}
\usepackage{etex} %for Windows
\usepackage[utf8]{inputenc}
\usepackage[english, russian]{babel} 
%\usepackage{microtype}			% Better interword spacing and additional kerning.
\usepackage{ellipsis}			% Adjusted space with \dots between two words.
\usepackage{graphicx}
\usepackage{pstricks}

\usepackage{xcolor}


\usepackage{changepage}

\usepackage{algorithm}
\usepackage{algpseudocode}
%\usepackage[]{algorithm2e}
%\usepackage{algorithmic}

%\usepackage{tcolorbox}


\usepackage{caption}
\usepackage{subcaption}
%\usepackage{stackengine}


\usepackage{tikz}
\usetikzlibrary{tikzmark,calc}
\usetikzlibrary{positioning, backgrounds}
\usetikzlibrary{arrows, chains, matrix, scopes, patterns, shapes, fit}
\usetikzlibrary{mindmap,trees,shadows}
\usetikzlibrary{decorations.pathreplacing}
%\usetikzlibrary{crypto.symbols}

\usepackage{pgfplots}

\pgfmathdeclarefunction{gauss}{2}{%
	\pgfmathparse{1/(#2*sqrt(2*pi))*exp(-((x-#1)^2)/(2*#2^2))}%
}


\tikzset{
	invisible/.style={opacity=0},
	visible on/.style={alt={#1{}{invisible}}},
	alt/.code args={<#1>#2#3}{%
		\alt<#1>{\pgfkeysalso{#2}}{\pgfkeysalso{#3}} % \pgfkeysalso doesn't change the path
	},
}

\newcommand\strikeout[2][]{%
	\begin{tabular}[b]{@{}c@{}} 
		\makebox(0,0)[cb]{{#1}} \\[-0.2\normalbaselineskip]
		\rlap{\color{Orange}\rule[0.5ex]{\widthof{#2}}{1.5pt}}#2
\end{tabular}}

\newcommand\Fontvi{\fontsize{11}{13.2}\selectfont}

\usepackage{listings} % for C++ code

\usepackage{braket}
%\usepackage[braket, qm]{qcircuit}



\usepackage[T1]{fontenc}
%\usepackage[sfdefault,scaled=.85]{FiraSans}
%\usepackage{newtxsf}
%\usepackage[nomap]{FiraMono}





\usefonttheme[onlymath]{serif}
\renewcommand\sfdefault{cmbr}

\renewcommand{\bfdefault}{sb}

\definecolor{CharCoalDark}{RGB}{13, 16, 19}
\definecolor{Orange}{RGB}{255, 165,0}
\definecolor{DarkOrange}{RGB}{255, 165,0}
\definecolor{LightSalmon}{RGB}{255, 160, 122}
\definecolor{LeafGreen}{RGB}{34, 139,  34}
\definecolor{Coral}{RGB}{255, 127, 80}
\definecolor{DarkTurquoise}{RGB}{0, 206, 209}

%\newtheorem{defRus}{Определение}
%\newtheorem{thmRus}{Теорема}
%s\newtheorem{corRus}{Следствие}


\setbeamercolor{background canvas}{bg=CharCoalDark}

\setbeamerfont{title}{series=\bfseries}
\setbeamercolor{title}{fg=Orange}
\setbeamercolor{section in toc}{fg=white}
\setbeamercolor{frametitle}{fg=Orange}
\setbeamercolor{normal text}{fg=white}
%\setbeamercolor{normal text}{fontsize=12pt}
\setbeamercolor{itemize item}{fg=Orange}
\setbeamercolor{enumerate item}{fg=Orange}
\setbeamercolor{enumerate item item}{fg=Orange}
\setbeamercolor{itemize item item}{fg=Orange}
\setbeamercolor{enumerate item}{fg=Orange}
\setbeamercolor{block title}{bg=DarkOrange,fg=white}
\setbeamerfont{block title}{series=\bfseries}

\setbeamertemplate{itemize item}[circle]
\setbeamertemplate{eumerate subitem}{\color{Orange}[$\checkmark$]}
\setbeamertemplate{itemize subitem}{\color{Orange}\Large$\textbullet$}
\setbeamertemplate{itemize subitem}{\color{Orange} \tiny $\blacksquare$}

% footnote without a marker
\newcommand\blfootnote[1]{%
	\begingroup
	\renewcommand\footnoterule{}
	\renewcommand\thefootnote{}\footnote{#1}%
	\addtocounter{footnote}{-1}%
	\endgroup
}

\newcommand*{\Scale}[2][4]{\scalebox{#1}{\ensuremath{#2}}}%

\newcommand\Item[1][]{%
	\ifx\relax#1\relax  \item \else \item[#1] \fi
	\abovedisplayskip=0pt\abovedisplayshortskip=0pt~\vspace*{-\baselineskip}}

\pgfdeclareradialshading{ring}{\pgfpoint{0cm}{0cm}}%
{rgb(0cm)=(1,1,1);
	rgb(0.7cm)=(1,1,1);
	rgb(0.719cm)=(1,1,1);
	rgb(0.72cm)=(0.975,0,0);
	rgb(0.9cm)=(1,1,1)}

\usepackage[absolute,overlay]{textpos} %to clip to a corner
\newcommand\FrameText[1]{%
	\begin{textblock*}{\paperwidth}(\textwidth-35pt, 10 pt)
		\raggedright #1\hspace{.5em}
\end{textblock*}}

\makeatletter
\let\save@measuring@true\measuring@true
\def\measuring@true{%
	\save@measuring@true
	\def\beamer@sortzero##1{\beamer@ifnextcharospec{\beamer@sortzeroread{##1}}{}}%
	\def\beamer@sortzeroread##1<##2>{}%
	\def\beamer@finalnospec{}%
}
\makeatother

\AtBeginSection[]
{
	\begin{frame}<beamer>
		\frametitle{Outline}
		\tableofcontents[currentsection]
	\end{frame}
}


%\institute{ENS Lyon}
\author{Elena Kirshanova \\ [10pt]
}
\titlegraphic{
	
	%\includegraphics[width=2.5cm]{erc_logo_gray}\hspace*{2.5cm}~%
	%\includegraphics[width=4.0cm]{ens_logo_gray}
}
\title{\Huge Message Authentication Code}

\date{ Course ``Information and Network Security'' \\ 	
	Lecture 5 \\ \today }


\setbeamertemplate{navigation symbols}{} %removes navigation

% proper highlightling of a code-snippet
\lstset{language=C++,
	keywordstyle=\color{magenta},
	stringstyle=\color{Goldenrod},
	commentstyle=\color{gray},
	breaklines=false,
	%morecomment=[l][\color{magenta}]{\#}
}

%\setlength{\parskip}{8pt}
\input{header} %all defs
\begin{document}
	
\begin{frame}
	\titlepage
\end{frame}


\begin{frame}{Confidentiality \& Integrity}
\LARGE 

\begin{itemize}
	\itemsep 1em
	\item Previous lectures dealt with {\color{Orange} confidentiality  }
	\item This lecture's goal:  {\color{Orange} integrity (целостность)}
\end{itemize}

\vspace{20pt}
\LARGE {\color{Orange}Primitive}: Message Authentication Code (MAC) \\[5pt]
Имитовставка или Код Аутентификации Сообщения


\end{frame}

\begin{frame}{MAC: definition}
\Large
Goal: send a message $m$ from Alice to Bob s.t.\ an adversary cannot modify $m$ without Bob noticing \\[10pt]
	\begin{center}
		\begin{tabular}{c c c c c}
			& Alice  & & Bob &  \\
			& \multirow{5}{*}{\includegraphics[scale=0.20]{Alice}} & &
			\multirow{5}{*}{\includegraphics[scale=0.20]{Bob}} &  \\
			&  & \Huge $\xrightarrow{\mkern25mu m \mkern25mu}$ & &  \\ 
			\pause
			\LARGE $k \in \keyS$ &   & &  & $k \in \keyS$ \\[10pt]
			\pause
			$\mactag \leftarrow S(k,m)$   & & &  &  \\ 
		\end{tabular}
	\end{center}
\end{frame}

\begin{frame}{MAC: definition}
\Large
Goal: send a message $m$ from Alice to Bob s.t.\ an adversary cannot modify $m$ without Bob noticing \\[5pt]
\begin{center}
	\begin{tabular}{c c c c c}
		& Alice  & & Bob &  \\
		& \multirow{5}{*}{\includegraphics[scale=0.18]{Alice}} & &
		\multirow{5}{*}{\includegraphics[scale=0.18]{Bob}} &  \\
		&  & \Huge $\xrightarrow{\mkern25mu m, \; \mactag \mkern25mu}$ & &  \\ 
		\LARGE $k \in \keyS$ &   & &  & $k \in \keyS$ \\[10pt]
		$\mactag \leftarrow S(k,m)$   & & &  & \pause   \\ [20pt] 
		\multicolumn{5}{r}{$V(k,m, \mactag) \in \{ 0, 1\}$ }  \\
		\multicolumn{5}{r}{$0-$ Accept, $1-$Reject}  
	\end{tabular}
\end{center}
\pause
\large
A {\color{Orange} Message Authentication Code } consists of three ppt algorithms:
\vspace{-2pt}
\begin{itemize}
	\item Key generation: $\KeyGen(1^\lambda): k \leftarrow \keyS$ \\[4pt]
	\item Tag generation: $S(k, m): \mactag \leftarrow \tagS$ \\[4pt]
	\item Tag verification: $V(k, m, \mactag): \{0, 1\}  $
\end{itemize}
\end{frame}

\begin{frame}{MAC: correctness and security}
\Large
A {\color{Orange} Message Authentication Code } consists of three ppt algorithms:
\vspace{-5pt}
\begin{itemize}
	\item Key generation: $\KeyGen(1^\lambda): k \leftarrow \keyS$ \\[4pt]
	\item Tag generation: $S(k, m): \mactag \leftarrow \tagS$ \\[4pt]
	\item Tag verification: $V(k, m, \mactag): \{0, 1\}  $  \\
\end{itemize}

\pause
\vspace{10pt}
{\color{Orange}Correcntess:} \LARGE 
$ V(k, m, S(k,m)) = \text{Accept} \; \forall \; m \in \mesS, k \in \keyS $

\vspace{10pt}
{\color{Orange}Security:}
For a ppt adversary $\adv$ who sees many (message, tag) pairs $\{(m_1, t_1), \ldots, (m_N, t_N) \}$ of its choice, $\adv$ cannot produce a new valid (message, tag) pair $(m,t)$
\[(m,t) \notin \{(m_1, t_1), \ldots, (m_N, t_N) \}\]

\vspace{10pt}

Therefore, secure tag length: $96, 128, 256 $ bits.
\end{frame}

\begin{frame}{Constructions of MACs}
\Large
\begin{enumerate}
	\itemsep1em
	\item A block-cipher  (AES, GOST) can be used to instantiate MAC for 16-bytes messages
	\item For longer messages:  \\
		-- CBC-MAC (integrity of bank transactions) \\ 
		Standards: ANSI, FIPS 186-3, GOST \\[5pt]
		-- HMAC (SSL, IPSec)
\end{enumerate}
\pause
\centering
\LARGE
\vspace{15pt}
This time: CBC-MAC
\end{frame}

\begin{frame}{CBC-MAC}
\Large
 $m = (m_1, m_2, m_3, ...)$, $F-$ a block cipher of block length $|m_i|$  
	\begin{figure}
		\includegraphics[width=0.95\textwidth]{CBC_MAC}
	\end{figure}
\vspace{-90pt}
\pause
\Large 
$k, k1$ - two different keys for $F $  shared by the parties\\ [10pt]
\pause
Verification executes the same routine \\ [10pt]
\pause
Double encryption of the last block is important \\
to prevent against insecure MACs (see PA4) \\[15pt]
The output $\mactag$ can be truncated to the desired bit length, but not less than $64$ bits.
\end{frame}

\begin{frame}{Compare CBC-MAC with CBC-mode for block-ciphers}
\vspace{10pt}
\begin{minipage}{0.18\textwidth} \Large CBC-MAC: \end{minipage}
\begin{minipage}{0.80\textwidth}
\begin{figure}
	\includegraphics[width=0.9\textwidth]{CBC_MAC} 
\end{figure}
\end{minipage}
\vspace{-10pt}
\begin{minipage}{0.18\textwidth} \Large CBC-mode: \end{minipage}
\begin{minipage}{0.80\textwidth}
	\begin{figure}
		\includegraphics[width=0.9\textwidth]{CBC_1} 
	\end{figure}
\end{minipage}

\end{frame}

\begin{frame}{Padding}
\Large
\begin{center}
What if $\mathbf{m} = (m_1, m_2, ...)$ is not a multiple of the block length?
\end{center}
Example of {\color{Orange} bad} padding: filling up with $0'$s.
\[m = (\star \star \star  )\] 
that has length $<$ block length. Padding of such message:
\[m_{\text{padded}}=(\star \star \star  \; 0 \ldots 0  ) \]. Then for another message $m'$:
\[S(k,m_{\text{padded}}) = S(k, m'=(\star \star \star  \; 0 \ldots 0  ))\]
For any $m$, $S(m)=S(m||0)$.

\pause
\vspace{10pt}
\LARGE \centering
For $\mathbf{m} \neq \mathbf{m'}$ we want $\mathbf{m}||\text{pad} \neq \mathbf{m'}||\text{pad}$. In particular, padding must be $1 \leftrightarrow 1$.
	
\end{frame}

\begin{frame}{Padding}
	\Large
	\begin{enumerate}
		\itemsep1em
		\item {\color{Orange} ISO.} Pad with $10\ldots0$. `1' indicates beginning of the padding. \\
		Add a dummy block if $|m| < $ block length
		\pause 
		\item {\color{Orange} NIST, GOST}  
		\begin{figure}
			\captionsetup[subfigure]{labelformat=empty}
			%\centering
			%\stackunder[5pt]{\includegraphics[width=.47\textwidth]{CBC_Mac_Padding}}{MRI-CGCM3}
			%\stackunder[5pt]{\includegraphics[width=.47\textwidth]{CBC_Mac_Padding_1}}{NorESM1-M}
			\begin{subfigure}{.5\textwidth}
				\centering
				\includegraphics[width=.98\textwidth]{CBC_Mac_Padding}
				\caption{$m_4$< block-length}
			\end{subfigure}%
		 \begin{subfigure}{.5\textwidth}
			\centering
			\includegraphics[width=.98\textwidth]{CBC_Mac_Padding_1}
			\caption{$m_4$= block-length}
		\end{subfigure}%
		\end{figure}
		\vspace{1em}
		Advantages: No dummy block, no additional application of $F$.
		 
	\end{enumerate}
\end{frame}

\begin{frame}{Some details on Programming Assignment 4}
\Large
Assume our tag is generated without the last encryption step for one-block message $m $ ($|m|=128$ bits).
\begin{figure}
	\includegraphics[width=0.8\textwidth]{Raw_CBC_Mac} 
\end{figure}

\vspace{-30pt}
Your task is to demonstrate an efficient tag forgery for such construction: by having $(m,t)$ - a valid (message, tag) pair, provide another valid pair $(m', t)$. \\
\emph{Hint:} $m'$ might consist of two blocks.
	
\end{frame}

\begin{frame}{Concluding slide}
\LARGE
	\begin{enumerate}
		\itemsep 1em
		\item Other types of MACs exist: Parallel Mac (PMAC), One-time MAC
		\item MACs are not Checksums! MACs {\color{Orange} require} secret keys
		\item DO NOT follow the Russian Wikipedia article on MACs
		\item For Russian project follow GOST'15 guidelines (link is on the course's webpage)
	\end{enumerate}
\vspace{10pt}
Next time: Message integrity with cryptographic hash functions
\end{frame}

\end{document}